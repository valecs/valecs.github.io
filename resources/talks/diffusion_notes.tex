\documentclass[12pt,a4paper]{article}
\usepackage{amsmath}
\usepackage{dsfont}
\usepackage{graphicx}
\usepackage[section]{placeins}
\usepackage[utf8]{inputenc}

\begin{document}

\title{Notes on Diffusion in Liquid Crystals}
\author{Layne Frechette}
\date{June 2014}
\maketitle

\section{Translational Diffusion}

One convenient expression for the translational diffusion constant in the isotropic phase is a Green-Kubo relation:

\begin{align*}
D=\frac{1}{3N}\sum_{i=1}^N\int_0^{\infty}\langle \vec{v}_i(0)\cdot \vec{v}_i(t)\rangle \, dt
\end{align*}

\noindent Using this formula, we can just numerically integrate the velocity autocorrelation function to get the diffusion coefficient. \\

\noindent In the nematic phase, where there is a preferred direction of alignment, we have a diffusion tensor which, when diagonalized, has two distinct components. These components, parallel and perpendicular to the director of the phase, are also given by Green-Kubo expressions:

\begin{align*}
D^{\parallel} &= \frac{1}{3N}\sum_{i=1}^N\int_0^{\infty}\langle \vec{v}_i(0)\cdot \hat{n}\hat{n} \cdot \vec{v}_i(t)\rangle \, dt\\
D^{\perp} &= \frac{1}{3N}\sum_{i=1}^N\int_0^{\infty}\langle \vec{v}_i(0)\cdot(\mathds{1}-\hat{n}\hat{n})\cdot \vec{v}_i(t)\rangle \, dt
\end{align*}

\noindent Here, $\hat{n}\hat{n}$ is the director dyad and $\mathds{1}$ is the unit tensor. For a nematic phase with $T^*\approx 1.00$, $\rho^*=0.34$, I calculated:

\begin{align*}
D^{*\parallel}&\approx 0.0333\\
D^{*\perp}&\approx 0.0180
\end{align*}

\noindent So, Gay-Berne molecules seem to be able to diffuse farther parallel to the director than perpendicular to the director.

\section{Rotational Diffusion}
One commonly used model for understanding rotational diffusion is the Debye model. In this model, a unit vector, describing the orientation of a molecule, diffuses on the surface of a sphere. The vector is assumed to take a random walk with small steps. We can write down a diffusion equation for this situation:
\begin{align*}
\nabla^2P(\psi, t)= D_R \frac{\partial P(\psi, t)}{\partial t}
\end{align*}
Here, $P(\psi, t)$ is the probability of diffusing an angle $\psi$ away from the initial orientation in a time $t$, and $D_R$ is the rotational diffusion coefficient. Solving this equation, subject to the condition that the unit vector is contrained to move on the surface of a sphere, yields the following expression for $P(\psi, t)$:

\begin{align*}
P(\psi, t) = \sum_{l=0}^{\infty}\left( \frac{2l+1}{2}\right) P_l(\cos{\psi})e^{-l(l+1)D_Rt}
\end{align*}
Where $P_l(\cos{\psi})$ is the (unassociated) $l$th order Legendre polynomial in $\cos{\psi}$.\\
Now, to find $D_R$, we consider the $l$th order reorientational time correlation function:
\begin{align*}
C^l(t) =  \frac{1}{N}\sum_{i=1}^N\langle P_l(\hat{\Omega}_i(0)\cdot \hat{\Omega}_i(t))\rangle
\end{align*}
Where we can identify $\hat{\Omega}_i(0)\cdot \hat{\Omega}_i(t) =\cos{\psi_i}$. In the Debye approximation, this function assumes a simple form:
\begin{align*}
C^l(t) = e^{-l(l+1)D_Rt}
\end{align*}
We can then define a correlation time: 
\begin{align*}
\tau_l = \int_{0}^{\infty}C^l(t) \, dt
\end{align*}
From which we can obtain the rotational diffusion coefficient:
\[
D_R = \frac{1}{l(l+1)\tau_l}
\]
Unfortunately, we cannot assume that the orientation vectors of molecules in the nematic phase are performing a random walk with short steps. The orientation vectors have a preferred direction of alignment (the director), and so will spend most of their time near the director. Large deviations from alignment with the director will likely require large-angle ``jumps.'' Thus, we cannot use the Debye approximation to calculate rotational diffusion coefficients in the nematic phase. An alternative is to calculate the rotational diffusion coefficient using mean-squared angular displacements:
\begin{align*}
D_R^{\alpha} &= \lim_{t\to\infty}\frac{1}{2tN}\sum_{i=1}^N\langle[\Delta \varphi_i^{\alpha}(t)]^2\rangle\\
\Delta \varphi_i^{\alpha}(t) &= \varphi_i^{\alpha}(t) - \varphi_i^{\alpha}(0) = \int_0^t \omega_i^{\alpha, b} \, dt
\end{align*}
Where $D_R^{\alpha}$ is the rotational diffusion coefficient in the $\alpha$th direction (relative to the molecules), and $\omega_i^{\alpha, b}$ is the $\alpha$th component of the angular velocity of the $i$th molecule, expressed in the body-fixed ($b$) frame. By subsituting the definition of $\Delta \varphi_i^{\alpha}(t)$ into the expression for $D_R$ (and performing a few manipulations), we arrive at a Green-Kubo expression for $D_R^{\alpha}$:
\[
D_R^{\alpha} = \frac{1}{N}\sum_{i=1}^N\int_0^{\infty}\langle \omega_i^{\alpha, b}(t)\omega_i^{\alpha, b}(0)\rangle \, dt
\]
In the nematic phase, $T^*\approx 1.00$, $\rho^*=0.34$, I obtained:
\begin{align*}
D_R^x &\approx 0.0129\\
D_R^y &\approx 0.0151
\end{align*}
These values are close, but not quite the same. I intend to determine whether this is due to numerical integration error, or whether it is actually a feature of the nematic phase.
\end{document}