\documentclass[letterpaper,12pt]{article} 
\title{Green Function for a small step in Paramagnetic spin model}
\author{Yan Zhao} 
\usepackage{amsmath} 
\usepackage{amssymb} 
\usepackage{url}
\usepackage{pifont}


\begin{document} 
\maketitle 
\textbf{Problem:}  

N spins forming a 1-D spin chain and they act independently.

$\sigma_i$ is the $i^\textsuperscript{th}$ spin and $\sigma_i$ can be 
$\left\{
  \begin{array}{lcr}
     1& up\\
     0& down
  \end{array}
  \right.$
  
The configuration of the system is 
 $\underset{\sim}{\sigma}=(\sigma_i,\sigma_2 , \cdots ,\sigma_N) $ .
 
Master equation
\[
\left.
\begin{aligned}
 \frac{d}{dt}P_t(\underset{\sim}{\sigma})&=- \underset{\underset{\sim}{\sigma}''}{\sum} W(\underset{\sim}{\sigma}\rightarrow\underset{\sim}{\sigma}'') P_t(\underset{\sim}{\sigma}) + \underset{\underset{\sim}{\sigma}'}{\sum} W( \underset{\sim}{\sigma}' \rightarrow \underset{\sim}{\sigma}) P_t(\underset{\sim}{\sigma}') \\
 &=-\underset{\underset{\sim}{\sigma}'}{\sum}\underset{\underset{\sim}{\sigma}''}{\sum}W(\underset{\sim}{\sigma}'\rightarrow\underset{\sim}{\sigma}'') P_t(\underset{\sim}{\sigma}')\delta_{\underset{\sim}{\sigma}',\underset{\sim}{\sigma}} + \underset{\underset{\sim}{\sigma}'}{\sum} W( \underset{\sim}{\sigma}' \rightarrow \underset{\sim}{\sigma}) P_t(\underset{\sim}{\sigma}')\\
 &=\underset{\underset{\sim}{\sigma}'}{\sum} \left( W( \underset{\sim}{\sigma}' \rightarrow \underset{\sim}{\sigma})-\delta_{\underset{\sim}{\sigma}',\underset{\sim}{\sigma}}\underset{\underset{\sim}{\sigma}''}{\sum}W(\underset{\sim}{\sigma}'\rightarrow\underset{\sim}{\sigma}'') \right) P_t(\underset{\sim}{\sigma}')\\
 &=\underset{\underset{\sim}{\sigma}'}{\sum} L(\underset{\sim}{\sigma}'\rightarrow\underset{\sim}{\sigma})P_t(\underset{\sim}{\sigma}') 
\end{aligned}
\right. 
\] 

By definition, 
\[
W( \underset{\sim}{\sigma} \rightarrow \underset{\sim}{\sigma})=0 \label {1}\tag{1}
\]

and 
\[
W( \underset{\sim}{\sigma}' \rightarrow \underset{\sim}{\sigma})=\sum\limits_{i=1}^N W_i(\underset{\sim}{\sigma}' \rightarrow \underset{\sim}{\sigma}) \label {2}\tag{2}
\]

and 
\[
W_i(\underset{\sim}{\sigma}' \rightarrow \underset{\sim}{\sigma})=\left( c\alpha\delta_{\sigma_i 1}  \delta_{\sigma'_i 0} +(1-c)\alpha \delta_{\sigma_i 0}  \delta_{\sigma'_i 1} \right) \underset{j\neq i}{\prod} \delta_{\sigma_j \sigma'_j} \label {3}\tag{3}
\]

And we have
 \[
 L(\underset{\sim}{\sigma}'\rightarrow\underset{\sim}{\sigma})=W( \underset{\sim}{\sigma}' \rightarrow \underset{\sim}{\sigma})-\delta_{\underset{\sim}{\sigma}',\underset{\sim}{\sigma}}\underset{\underset{\sim}{\sigma}''}{\sum}W(\underset{\sim}{\sigma}'\rightarrow\underset{\sim}{\sigma}'')\label {4}\tag{4}
 \]
 
 Then we can substitute \eqref{2} and \eqref{3} into \eqref{4} , which will give us
 
 \[
 L(\underset{\sim}{\sigma}'\rightarrow\underset{\sim}{\sigma})=\sum\limits_{i=1}^N \left[ c\alpha \delta_{\sigma'_i 0} (\delta_{\sigma_i 1}-\delta_{\sigma_i \sigma'_i })+(1-c)\alpha \delta_{\sigma'_i 1}(\delta_{\sigma_i 0}-\delta_{\sigma_i \sigma'_i}) \right]  \underset{j \neq i}{\prod}\delta_{\sigma_i \sigma'_i}
 \label{5} \tag{5}
 \]

Let's go back to the master equation, by setting $dt=\epsilon$

\[
\frac{P_{t+\epsilon}(\underset{\sim}{\sigma})-P_t(\underset{\sim}{\sigma})}{\epsilon}=\underset{\underset{\sim}{\sigma}'}{\sum} L(\underset{\sim}{\sigma}'\rightarrow\underset{\sim}{\sigma})P_t(\underset{\sim}{\sigma}')
\]

Then
\[
\left.
\begin{aligned}
P_{t+\epsilon}(\underset{\sim}{\sigma})&=P_t(\underset{\sim}{\sigma})+\underset{\underset{\sim}{\sigma}'}{\sum}\epsilon L(\underset{\sim}{\sigma}'\rightarrow\underset{\sim}{\sigma})P_t(\underset{\sim}{\sigma}')\\
&=\underset{\underset{\sim}{\sigma}'}{\sum} \left[  \delta_{\underset{\sim}{\sigma} \underset{\sim}{\sigma}'}+\epsilon L(\underset{\sim}{\sigma}'\rightarrow\underset{\sim}{\sigma}) \right] P_t(\underset{\sim}{\sigma}')
\end{aligned}
\right.
\] 
 
Compared with 
\[
P_{t+\epsilon}(\underset{\sim}{\sigma})=\underset{\underset{\sim}{\sigma}'}{\sum} G(\underset{\sim}{\sigma}' \rightarrow \underset{\sim}{\sigma} ; \epsilon)P_t(\underset{\sim}{\sigma}')
\]

So the green function for this small step $(t \rightarrow t+\epsilon )$  is
\[
\begin{aligned}
 G(\underset{\sim}{\sigma}' \rightarrow \underset{\sim}{\sigma} ; \epsilon)&=\delta_{\underset{\sim}{\sigma} \underset{\sim}{\sigma}'}+\epsilon L(\underset{\sim}{\sigma}'\rightarrow\underset{\sim}{\sigma})\\ 
 &=\sum\limits_{i=1}^N \frac{1}{N} \delta_{\sigma_i \sigma'_i} \underset{j \neq i}{\prod}\delta_{\sigma_i \sigma'_i} +\epsilon L(\underset{\sim}{\sigma}'\rightarrow\underset{\sim}{\sigma})\\
 &=\sum\limits_{i=1}^N \left[ \frac{1}{N} \delta_{\sigma_i \sigma'_i} +c\epsilon\alpha \delta_{\sigma'_i 0} (\delta_{\sigma_i 1}-\delta_{\sigma_i \sigma'_i })+(1-c)\epsilon\alpha \delta_{\sigma'_i 1}(\delta_{\sigma_i 0}-\delta_{\sigma_i \sigma'_i})\right] \underset{j \neq i}{\prod}\delta_{\sigma_i \sigma'_i}
\end{aligned}
 \]

If $a$ and $b$ are just $1$ or $0$, we can write down

\[
\delta_{ab}=1+2ab-a-b.
\]

By using this trick, we can simplify the green function. Finally, we will have

\[
\left.
\begin{aligned}
&G(\underset{\sim}{\sigma}' \rightarrow \underset{\sim}{\sigma} ; \epsilon)\\
=\sum\limits_{i=1}^N& \left[ \left(\frac{1}{N}-c\epsilon\alpha \right) + \left( \frac{2}{N}-2\epsilon\alpha \right) \sigma_i\sigma'_i - \left( \frac{1}{N} - 2c\epsilon \alpha \right) \sigma_i - \left( \frac{1}{N}- \epsilon\alpha \right) \sigma'_i \right]  \\
&\underset{j \neq i}{\prod} \left(1+2\sigma_i\sigma'_i-\sigma_i-\sigma'_i \right) \\
=\sum\limits_{i=1}^N &(A+B\sigma_i\sigma'_i-C\sigma_i-D\sigma'_i) \underset{j \neq i}{\prod} (1+2\sigma_i\sigma'_i-\sigma_i-\sigma'_i)\\
where\\
&A=\frac{1}{N}-c\epsilon\alpha\\
&B=\frac{2}{N}-2\epsilon\alpha\\
&C=\frac{1}{N} - 2c\epsilon \alpha\\
&D=\frac{1}{N}- \epsilon\alpha\\
\end{aligned}
\right.
\]


Now I am still trying to change the form of this green function so that I could easily calculate the green function for a given path. And probably I could use other tricks instead of $\delta_{ab}=1+2ab-a-b$. Please tell me if you have any ideas or questions, we can discuss and I really appreciate it!
\bigskip

\noindent\textbf{References}

[1]  R. van Zon and J. Schofield, J. Chem. Phys. 122, 194502 (2005)

\end{document}