\documentclass[letterpaper,12pt]{article}

\usepackage{times}
\usepackage[]{hyperref}
\usepackage[margin=1in]{geometry}
\usepackage[section]{placeins}
\usepackage{graphicx}
\usepackage{amsmath,amsfonts,amssymb}
\usepackage[version=3]{mhchem}
\usepackage{minted}
\usepackage{mathrsfs} % formal script font; use \mathscr{...}
\usepackage[titletoc]{appendix} % adds ``Appendix'' in TOC
\usepackage{datetime} % gives \currenttime
\usdate % keeps standadard \LaTeX date format for \today
\usepackage{gensymb}
\usepackage{bbm}% for \mathbbm % blackboard bold
\usepackage{mathtools} % gives \substack

%\setcounter{tocdepth}{2}

\title{Path Integral Motivation for the Geodesic View}
\date{Modified: \today\ at \currenttime}
\author{Vale Cofer-Shabica}

\renewcommand{\vec}[1]{\mathbf{#1}}
\newcommand{\mat}[1]{\;\textrm{\textbf{#1}}\,}
%\newcommand{\mat}[1]{\mbox{\normalsize $\,\mathtt{#1}\,$}}
%\newcommand{\mat}[1]{\,\mathtt{#1}\,}
\newcommand{\trans}[1]{{#1}^{\mathsf{T}}}
\newcommand{\norm}[1]{\left\lVert#1\right\rVert}

\newcommand{\soop}[1]{\textsuperscript{#1}}
%\renewcommand{\th}{\textsuperscript{th}}
\renewcommand{\th}{th}

\renewcommand{\refeq}[1]{eq. \ref{eqn:#1}}

\newcommand{\imsize}{4in}

\definecolor{mintbg}{rgb}{0.95,0.95,0.95}

\newcommand{\notesym}{\textsuperscript{*}}

\begin{document}
\maketitle

\abstract{A derivation and discussion of Wang and Stratt's \cite{wang:2007:geodesics} path-integral expression for the diffusion propagator. We follow Weigel \cite{wiegel:1986} and note some of the implications of the expression.}

\section*{Introduction}
We wish to understand the following equation (2.1) from Wang and Stratt's second paper of 2007 \cite{wang:2007:geodesics}:
\begin{equation}
G(\vec{R}_{0} \to \vec{R} , t) = \int_{\vec{R}_{0},0}^{\vec{R},t} \mathscr{D} [\vec{R}(\tau)]\exp\left[ -\frac{1}{4D} \int_{0}^{t} {\left(\frac{d\vec{R}}{d\tau} \right)}^{2} d\tau\right] \;,
\end{equation}
which gives the Green's function for a system under free diffusion (no forces) as a path integral. From this point, the authors argue that the dynamics of chemical systems are well captured by the paths which are ``shortest''. This observation has  motivated much of the group's work during and since 2007.

This document contains two parts: a derivation of the above expression and a discussion which motivates it as well as leads from it to the notion of shortest paths as dominant.

\section*{Derivation}
\emph{This section shamelessly follows the first chapter of Weigel's excellent introduction to path integrals \cite{wiegel:1986}.}  In the absence of external forces, diffusion is a process characterized by the following dynamical relation:
\begin{equation}
\frac{d}{dt} G(\vec{R} , t) = D \nabla^{2} G(\vec{R} , t) \; ,
\end{equation}
where $G(\vec{R} , t)$ is the probability of finding a system in configuration $\vec{R}$ at time $t$ and $D$ is a constant called the diffusion constant. We take up the question of why concern ourselves with (free) diffusion in the next section.

With the initial condition $G(\vec{R}, t = 0) = \delta(\vec{R}- \vec{R}_{0})$, one can find the solution for a Cartesian system of $\alpha$ degrees of freedom to be:
\begin{equation} \label{eqn:gSoln}
G(\vec{R}_{0} \rightarrow \vec{R} , t) = (4\pi D t)^{-\alpha/2}\exp\left[ -\frac{(\vec{R} - \vec{R}_0)^2}{4Dt}\right] \; ,
\end{equation}
as can be verified by substitution. Our notation is to indicate the probability of diffusing from $\vec{R}_0$ to $\vec{R}$ in a time $t$.

Suppose we are interested in  a particular path between $\vec{R}_0$ and $\vec{R}$. We could represent such a path by discretizing it into $(N+1)$ bits of time, each of duration $\varepsilon$, such that: $(N+1)\varepsilon=t$. We can then define the following: $ t_{i} \equiv \varepsilon \cdot i $, and  $t_{N+1} \equiv t$, and $\vec{R}_{i}$ as the configuration of the system at time $t_i$. This procedure is illustrated in figure \ref{fig:path}.

\begin{figure}[H]
\begin{center}
\includegraphics[width=.8\textwidth]{weigel1986}
\caption{Discretizing a Path}{\label{fig:path}\emph{Figure modified from} \cite{wiegel:1986}.}
\end{center}
\end{figure}

We can also compute the probability of tracing such a path. Because they are independent,  it is the probability of propagating from $\vec{R}_0$ to $\vec{R}_1$ in time $\varepsilon$ multiplied by the probability of propagating from $\vec{R}_1$ to $\vec{R}_2$ in time $\varepsilon$ and so on. By inserting \refeq{gSoln} we have:
\begin{align} \label{eqn:gDiscrete}
G(\vec{R} \rightarrow \vec{R}_{0} , t ; \{\vec{R}_1, \vec{R}_2. \ldots \vec{R}_N\})  &= \prod_{i=0}^{N} G(\vec{R}_{i} \rightarrow \vec{R}_{i+1}, \varepsilon) \\
&=  (4\pi D \varepsilon)^{-\alpha(N+1)/2}\exp\left[ - \sum_{i=0}^{N} \frac{(\vec{R}_{i+1} - \vec{R}_i)^2}{4D\varepsilon}\right] \; ,
\end{align}
which gives the probability of diffusing from $\vec{R}_{0}$ to $\vec{R}$ in a time $t$ via the $N$ ordered points $\{\vec{R}_1, \vec{R}_2. \ldots \vec{R}_N\}$. We recover our original expression (\refeq{gSoln}) by integrating \refeq{gDiscrete} over the domain of each $\vec{R}_{i}$:
\begin{align}
G(\vec{R} \rightarrow \vec{R}_{0} , t) &= \int d\vec{R}_1 \int d\vec{R}_2 \ldots \int d\vec{R}_N  G(\vec{R} \rightarrow \vec{R}_{0} , t ; \{\vec{R}_1, \vec{R}_2. \ldots \vec{R}_N\}) 
\end{align}

In the continuous limit, $N \to \infty$ and $\varepsilon \to 0$; focusing for the moment on the exponential, we have:
\begin{equation}
\begin{aligned}
\lim_{\substack{\varepsilon \to 0 \\ N \to \infty}} &\exp\left[ - \sum_{i=0}^{N} \frac{(\vec{R}_{i+1} - \vec{R}_i)^2}{4D\varepsilon}\right] =\\ 
\lim_{\substack{\varepsilon \to 0 \\ N \to \infty}}  &\exp\left[ - \frac{1}{4D}\sum_{i=0}^{N} {\left( \frac{\vec{R}_{i+1} - \vec{R}_i}{\varepsilon} \right)}^{2} \varepsilon \right] =\\
&\exp\left[ - \frac{1}{4D} \int_{0}^{t} \left(\frac{d\vec{R}}{d\tau}\right)^{2} d\tau\right] \; ,
\end{aligned}
\end{equation}
where $\vec{R}(\tau)$ is a continuous function on $[0,t]$specifying the path. Now we can write:
\begin{equation}
\begin{aligned}
G(\vec{R} \rightarrow \vec{R}_{0} , t) &=\\
\lim_{\substack{\varepsilon \to 0 \\ N \to \infty}} (4\pi D \varepsilon)^{-\alpha(N+1)/2} &\int d\vec{R}_1 \int d\vec{R}_2 \ldots \int d\vec{R}_N  \exp\left[ - \frac{1}{4D} \int_{0}^{t} \left(\frac{d\vec{R}}{d\tau}\right)^{2} d\tau\right]
\end{aligned}
\end{equation}
To express this more compactly, we define the following operator:
\begin{equation}
\int_{\vec{R}_{0},0}^{\vec{R},t} \mathscr{D}[\vec{R}(\tau)] \equiv \lim_{\substack{\varepsilon \to 0 \\ N \to \infty}} (4\pi D \varepsilon)^{-\alpha(N+1)/2} \int d\vec{R}_1 \int d\vec{R}_2 \ldots \int d\vec{R}_N \; ,
\end{equation}
which explicitly specifies the boundary values: $\vec{R}_{0}$ at $t=0$ and $\vec{R}$ at time $t$. With this notation in hand, we arrive at our original equation:
\begin{equation} \label{eqn:gFinal}
G(\vec{R} \rightarrow \vec{R}_{0} , t) = \int_{\vec{R}_{0},0}^{\vec{R},t} \mathscr{D} [\vec{R}(\tau)]\exp\left[ -\frac{1}{4D} \int_{0}^{t} {\left(\frac{d\vec{R}}{d\tau} \right)}^{2} d\tau\right] \;,
\end{equation}
which expresses the diffusion propagator as an integral over the space of all possible functions connecting the boundary values.

\section*{Discussion}
\subsection*{Why Diffusion?}
Why do we concern ourselves with diffusion, and in particular, free diffusion? Why do we preform all of this work when we already \emph{know} the solution? In the case of liquids, we examine diffusion because it is the mechanism of their molecular dynamics---the changes in structure are diffusive. In small molecule systems, we rely on two observations. First, and somewhat crassly, this method provides us with a rational filtering technique---a way to select the inherent dynamics of a system under study. Second, while the notion of diffusion is not applicable to the atomic motions of a small molecule, it is not unreasonable to consider the diffusion of energy between different parts of the system.

Analysis in the absence of external forces allows us to treat complicated systems from another perspective. We \emph{could} solve the diffusion equation in the presence of a complicated potential, but this is quite challenging and not particularly general. Treating free diffusion, on the other hand, is easy, but there's no such thing as a free lunch: we require our system to remain in field free regions during every infinitesimal step. Practically, this amounts to skirting the boundaries of obstacles and not interacting with them directly.

It turns out that an algorithm to deal with complicated boundary conditions is not only tractable, but far more general than a method to solve the diffusion equation in a broad class of chemical systems. The algorithm described in \cite{wang:2007:geodesics} is an example, and the workhorse of the group's studies.

\subsection*{Why \emph{Shortest} Paths?}
In their paper \cite{wang:2007:geodesics}, Wang and Stratt argue that \refeq{gFinal} implies that the trajectories which dominate the path integral are those with the shortest length. However, the form of \refeq{gFinal} contains no expression for the length.  The key observation is the following: the extremization of any quantity of the form:
\begin{equation}
I = \int f(x)dx
\end{equation}
can also be effected by extremizing 
\begin{equation}
I' = \int H(f(x))dx
\end{equation}
where $H(x)$ is a function with a strictly positive derivative and  whose domain includes the range of $f(x)$. Thus, minimizing
\begin{equation}
\int_{0}^{t} {\left(\frac{d\vec{R}}{d\tau} \right)}^{2} d\tau
\end{equation}
can be achieved by minimizing 
\begin{equation} \label{eqn:length}
\int_{0}^{t} \sqrt{{\left(\frac{d\vec{R}}{d\tau} \right)}^{2}} d\tau \; .
\end{equation}
The integral in  \refeq{length} can be reduced to
\begin{equation}
\int_{0}^{t} \sqrt{d\vec{R} \cdot d\vec{R}} = \int_{0}^{t} \norm{d\vec{R}} \; ,
\end{equation}
which is clearly the length of the path.

\bibliographystyle{JAmChemSoc}
\bibliography{../bibs/general,../bibs/roaming,../bibs/geodesics,../bibs/semiclassical,../bibs/self}

\end{document}
