\documentclass[letterpaper,12pt]{article}

\usepackage{times}
\usepackage[]{hyperref}
\usepackage[margin=1in]{geometry}
\usepackage[section]{placeins}
\usepackage{graphicx}
\usepackage{amsmath,amsfonts,amssymb}
\usepackage[version=3]{mhchem}
\usepackage{minted}
\usepackage{mathrsfs} % formal script font; use \mathscr{...}
\usepackage[titletoc]{appendix} % adds ``Appendix'' in TOC
\usepackage{datetime} % gives \currenttime
\usdate % keeps standadard \LaTeX date format for \today
\usepackage{gensymb}
\usepackage{bbm}% for \mathbbm % blackboard bold
\usepackage{mathtools} % gives \substack

%\setcounter{tocdepth}{2}

\title{Analyzing Geodesics:\\Beyond $\Delta R / g$}
\date{Modified: \today\ at \currenttime}
\author{Vale Cofer-Shabica}

\renewcommand{\vec}[1]{\mathbf{#1}}
\newcommand{\mat}[1]{\;\textrm{\textbf{#1}}\,}
%\newcommand{\mat}[1]{\mbox{\normalsize $\,\mathtt{#1}\,$}}
%\newcommand{\mat}[1]{\,\mathtt{#1}\,}
\newcommand{\trans}[1]{{#1}^{\mathsf{T}}}
\newcommand{\norm}[1]{\left\lVert#1\right\rVert}

\newcommand{\soop}[1]{\textsuperscript{#1}}
%\renewcommand{\th}{\textsuperscript{th}}
\renewcommand{\th}{th}

\renewcommand{\refeq}[1]{eq. \ref{eqn:#1}}

\newcommand{\imsize}{4in}

\definecolor{mintbg}{rgb}{0.95,0.95,0.95}

\newcommand{\notesym}{\textsuperscript{*}}

\begin{document}
\maketitle

\abstract{I discuss some of the methods used to analyze geodesic pathways computed on the formaldehyde potential energy surface. Chief among them are the decomposition of the total path length into different kinematic lengths and a novel method for the estimation of the path length from first principles in a small molecule system.}

\section*{Introduction}
The formalism discussed last time points to a powerful technique for understanding the dynamics of chemical systems. Our group has developed an algorithm for computing the geodesics (or shortest paths) for arbitrary, smooth potential surfaces. But once we have them, how do we extract information from them?

One of the first measures we used was $\Delta R/g$: the ratio of the Euclidean distance to length of the geodesic between points in configuration space. This metric is a quantitative predictor of the diffusion constant \cite{wang:2007:geodesics}:
\begin{equation*}
\frac{D}{D_{0}} = \lim_{\Delta R \to \infty} \overline{(\Delta R/g)^2}
\end{equation*}
Since that paper, new systems have called for new analyses. Here, I review the method of decomposing the kinematic length into lengths relating to different momenta (and corresponding to motions of various degrees of freedom).


\section{Decomposing the Geodesic}
Having computed some tens of thousands of geodesics, we would like to be able to analyze them in aggregate. Given that the geodesics are characterized by a minimum kinematic length, measuring the lengths of the paths is a natural place to start. The total kinematic length of a path is found as follows:
\begin{equation} \label{eqn:ltotal0}
\ell_{Total} = \int d\tau \left[ 2 \cdot T(\tau) \right]^{1/2}
\end{equation}
where $T$ is the kinetic energy, $\tau$ is a progress variable on $[0,1]$ and the limits of integration are over the entire path.

As the kinetic energy is partitioned between many  degrees of freedom, we can interrogate kinematic lengths corresponding to a subset of them by constructing the appropriate kinetic energy (\emph{e.g.}: rotation or vibration) and integrating as before. In doing so we can probe the contributions of different kinds of motion to the inherent dynamics.

%% In general, we can construct arbitrary kinematic lengths provided that we can construct a kinetic energy, $T_{\alpha}$, such that:
%% \begin{align*}
%% H &= \sum_{\alpha} T_{\alpha} + V \\
%% T_{\alpha} &= T_{\alpha}(\vec{x}, \dot{\vec{x}}) \\
%% %\textrm{and}\; T &\neq f
%% \end{align*}
%% \textbf{Other requirements on this function? I'm certain there are.}


\paragraph{With respect to integration,}\hspace{-1em} I perform all of the following integrals using a modified form of trapezoidal rule with unequally spaced abscissas \cite[eq. 25.4.1]{as:1964}. The expression is:
\begin{equation}\label{eqn:integrals}
\int_{\tau_{0}}^{\tau_{n}} d\tau f(\tau)  \approx \sum_{i=0}^{n-1} {\delta \tau}_{i} f\left({\tau}_{i}\right)
\end{equation}
In all expressions for the kinematic length, ${\delta \tau}_{i}$ will drop out; this is shown explicitly only in the case of the total kinematic length, which follows.


\subsection{Total Kinematic Length}
Picking up from \refeq{ltotal0}, we insert $T = \frac{1}{2} \dot{\vec{x}} \mat{M} \trans{\dot{\vec{x}}}$ and have:
\begin{equation}\label{eqn:ltotal1}
\ell_{Total} = \int d\tau \left[ \dot{\vec{x}} \mat{M} \trans{\dot{\vec{x}}} \right]^{1/2}
\end{equation}
where  $\dot{\vec{x}} = \frac{d\vec{x}}{d\tau}$ is the rate of change of the configuration space vector $\vec{x}$ and $\mat{M}$ is the diagonal mass matrix:
\begin{equation}\label{eqn:massmat}
\mat{M} = 
\begin{pmatrix}
m_1  &     &     &     &     &        &     \\
     & m_1 &     &     &     &        &     \\
     &     & m_1 &     &     &        &     \\
     &     &     & m_2 &     &        &     \\
     &     &     &     &     & \ddots &     \\
     &     &     &     &     &        & m_N 
\end{pmatrix}
\end{equation}
Where $m_i$ is the mass of the $i$\th\ particle in the $N$-particle system. A see table \ref{tab:masses} for the masses relevant to formaldehyde. Expressing our integral (\refeq{ltotal1}) as a sum (via \refeq{integrals}) yields:
\begin{align}
\ell_{Total}  &= \sum_{i=0}^{n-1} {\delta \tau}_{i} \left[ \dot{\vec{x}}_{i} \mat{M} \trans{\dot{\vec{x}}}_{i} \right]^{1/2} \\
&= \sum_{i=0}^{n-1} {\delta \tau}_{i} \left[ \frac{\delta \vec{x}_{i}}{{\delta \tau}_{i}} \mat{M} \trans{\frac{\delta \vec{x}_{i}}{{\delta \tau}_{i}}}\right]^{1/2} \\
&= \sum_{i=0}^{n-1}  \left[ {\delta \vec{x}^{\;(i)}} \mat{M} \trans{\delta \vec{x}^{\;(i)}}\right]^{1/2}
\end{align}
where $\delta \vec{x}^{\;(i)} = \vec{x}^{\;(i+1)} - \vec{x}^{\;(i)}$.

\begin{table} [h]
\begin{center}
\begin{tabular}{c c}
Element & Mass / $m_e$\notesym \\
\hline 
\ce{H} & 1837.15\\
\ce{C} & 21874.66\\
\ce{O} & 29156.95
\end{tabular}
\end{center}
\caption{Atomic masses}{\label{tab:masses}\notesym: $m_e$ is the rest mass of the electron.}
\end{table}

\subsection{Vibrational Kinematic Length}
To find the component of the kinematic length resulting from vibration, we construct the vibrational kinetic energy for 2 particles, $\alpha$ and $\beta$:
\begin{equation}\label{eqn:tvib}
T_{Vib} = \frac{1}{2} \mu \dot{\vec{r}}^{\;2}
\end{equation}
Where $\vec{r} = \vec{r}_{\alpha} - \vec{r}_{\beta}$ is the separation between the centers and $\mu = \frac{m_{\alpha} \cdot m_{\beta}}{m_{\alpha}+m_{\beta}}$ is the reduced mass. In analogy with \refeq{ltotal0}, we insert \refeq{tvib} into the expression for the length (\refeq{ltotal0}) and have:
\begin{align}\label{eqn:lvib}
\ell_{Vib} &= \int d\tau \left[\mu \dot{\vec{r}}^{\;2}\right]^{1/2} \\
&= \sqrt{\mu} \int d\tau  \norm{\dot{\vec{r}}}
\end{align}
The resulting sum is:
\begin{equation}
\ell_{Vib} = \sqrt{\mu} \sum_{i=0}^{n-1}  \norm{\delta\vec{r}^{\;(i)}}
\end{equation}
where  $\delta \vec{r}^{\;(i)} = \vec{r}^{\;(i+1)} - \vec{r}^{\;(i)}$.

\subsection{Rotational Kinematic Length}
We can perform a similar calculation for the rotational degrees of freedom. The rotational kinetic energy for a pair of particles, $\alpha$ and $\beta$, is:
\begin{align}
T_{Rot} &= \frac{1}{2}I\vec{\omega}^{\,2} \\
&=\frac{1}{2} \mu r^{2} \left( \frac{d\hat{r}}{d\tau}  \right)^{2}
\end{align}
where $r = \norm{\vec{r}}$ is the inter-center separation,  $\mu = \frac{m_{\alpha} \cdot m_{\beta}}{m_{\alpha}+m_{\beta}}$ is the reduced mass,  and $\hat{r} = \frac{\vec{r}}{\norm{\vec{r}}} = \frac{\vec{r}}{r}$ is the unit orientation vector (therefore: $\vec{\omega}^{\,2} = \left( \frac{d\hat{r}}{d\tau}  \right)^{2}$ ). Again, we insert the expression for kinetic energy into \refeq{ltotal0} and have:
\begin{align}
\ell_{Rot} = \sqrt{\mu} \int d\tau \left[r^{2} \left( \frac{d\hat{r}}{d\tau}  \right)^{2}\right]^{1/2}
\end{align}
Now we compute:
\begin{align}
\frac{d\hat{r}}{d\tau} &= \frac{d}{d\tau} \left[ \frac{\vec{r}}{\norm{\vec{r}}} \right] \\
 &= \frac{\dot{\vec{r}}}{r} - \frac{\vec{r}}{r^2}\frac{d}{d\tau}\left[ \norm{\vec{r}}\right] \\
 &= \frac{\dot{\vec{r}}}{r} - \frac{\vec{r}}{r^2} \left( \frac{\vec{r} \cdot \dot{\vec{r}}}{r}\right)
\end{align}
With some re-arrangement, we arrive at a more compact form:
\begin{align}
\frac{d\hat{r}}{d\tau}  &= \frac{\dot{\vec{r}}}{r} - \hat{r} \left(\hat{r} \cdot \frac{\dot{\vec{r}}}{r}\right) \\
 &= \frac{1}{r} \left[ \dot{\vec{r}} - \hat{r}\left( \hat{r} \cdot \dot{\vec{r}} \right)\right] \\
 &= \frac{\dot{\vec{r}}}{r} \cdot \left[ \mathbbm{1} - \hat{r}\hat{r} \right]
\end{align}
where $\mathbbm{1}$ is the identity and we have used the dyadic identity $\vec{x}(\vec{x}\cdot\vec{y}) = \vec{y}\cdot\vec{x}\vec{x}$. See \cite[2014.05.22:148]{dvcs:labbook1} for derivation.

With $\frac{d\hat{r}}{d\tau}$ in hand, we can compute the argument to the square root in our integral:
\begin{align}
r^{2} \left( \frac{d\hat{r}}{d\tau} \right) ^2 &= r^2 \left[ \frac{\dot{\vec{r}}}{r} \cdot \left( \mat{I} - \hat{r}\hat{r} \right) \right] \cdot \trans{\left[ \frac{\dot{\vec{r}}}{r} \cdot \left( \mat{I} - \hat{r}\hat{r} \right) \right]} \\
&= \dot{\vec{r}} \cdot \left( \mat{I} - \hat{r}\hat{r} \right) \cdot \trans{\left( \mat{I} - \hat{r}\hat{r} \right)} \cdot \trans{\dot{\vec{r}}} \\
&= \dot{\vec{r}} \cdot \left( \mat{I} - \hat{r}\hat{r} \right)  \cdot \trans{\dot{\vec{r}}}
\end{align}
The tensor terms in the penultimate step can be reduced by simple distribution and the identity: $(\hat{x}\hat{x})^2 = \hat{x}\hat{x}$, which holds for all unit vectors $\hat{x}$. See \cite[2014.05.22:148]{dvcs:labbook1} for derivation.

Algebra out of the way, we can write a final expression for the rotational kinematic length:
\begin{equation}
\ell_{Rot} = \sqrt{\mu} \int d\tau \left[ \dot{\vec{r}} \cdot \left( \mat{I} - \hat{r}\hat{r} \right) \cdot \trans{\dot{\vec{r}}} \right]^{1/2}
\end{equation}
which is then reduced to the sum:
\begin{equation}
\ell_{Rot} = \sqrt{\mu} \sum_{i=0}^{n-1}  \left[ \delta\vec{r}^{\;(i)} \cdot \left( \mat{I} - \hat{r}^{\,(i)}\hat{r}^{\,(i)} \right) \cdot \trans{\delta\vec{r}^{\;(i)}}\right]^{1/2}
\end{equation}
where  $\delta \vec{r}^{\;(i)} = \vec{r}^{\;(i+1)} - \vec{r}^{\;(i)}$ and $\hat{r}^{\,(i)} = \vec{r}^{\;(i)} / r^{(i)}$. 

\section{Estimating the Length of the Geodesic}
To verify our calculations it is desirable to make an analytic estimate of the sizes of the kinematic lengths involved. Here we use a very simple model to make such an estimate.

\newcommand{\co}{_{\ce{CO}}}
\newcommand{\hh}{_{\ce{H2}}}
\newcommand{\hhco}{_{\ce{H2}-\ce{CO}}}

\paragraph{Total Kinematic Length:}
We conceive of our system as having only 2 components: a mass of $m\hh = 2\cdot m_{\ce{H}}$ located at $\vec{r}\hh$ and a mass of $m\co = m_{\ce{C}} + m_{\ce{O}}$ located at $\vec{r}\co$. Further we will content ourselves to deal only with the translational motions of these fragments and therefore restrict ourselves to a line. Finally, we assume that all motion is barrier-free---that no potential obstacles intervene, or that for a straight-line path,  $V(\vec{x}) \le E_L$.

Are these reasonable simplifications? They are if the contribution to the kinematic length from translation is much greater than from internal rotation or vibration of \ce{H2} and \ce{CO}. In particular, we consider only direct paths, which are relatively unhindered. Therefore I expect these conditions to yield an estimate which is of the proper order of magnitude but at the lower edge of the distribution, perhaps even a lower bound. 

The kinetic energy of our system is:
\begin{equation}
T = \frac{1}{2}m\hh \dot{\vec{r}}^{\;2}\hh + \frac{1}{2}m\co \dot{\vec{r}}^{\;2}\co
\end{equation}
Recalling our expression for kinematic length (\refeq{ltotal0}), we need to find an expression for $2T$:
\begin{equation}
2T = m\hh \dot{\vec{r}}^{\;2}\hh + m\co \dot{\vec{r}}^{\;2}\co
\end{equation}
If we assume constant velocity during dissociation and a total time $t$, we can write:
\begin{equation}\label{eqn:T2-0}
2T = m\hh \left|\frac{\Delta \vec{r}\hh}{t} \right|^2 + m\co \left|\frac{\Delta \vec{r}\co}{t} \right|^2
\end{equation}
where $\Delta \vec{r}\co$ and $\Delta \vec{r}\hh$ are the total distances traveled by \ce{CO} and \ce{H2} respectively. Fixing our total momentum to zero allows us to write:
\begin{equation}
0=\vec{p}\hh+\vec{p}\co
\end{equation}
which implies:
\begin{equation}
m\co \frac{\Delta \vec{r}\co}{t} = -m\hh \frac{\Delta \vec{r}\hh}{t}
\end{equation}
and therefore:
\begin{equation}\label{eqn:rcoASrhh}
\Delta \vec{r}\co = -\Delta \vec{r}\hh \frac{m\hh}{m\co}
\end{equation}
Inserting this into our expression for $T2$ (\refeq{T2-0}) yields:
\begin{equation}\label{eqn:T2-1}
2T = m\hh {\left( \frac{\Delta \vec{r}\hh}{t}\right)}^2 \left(1 + \frac{m\hh}{m\co} \right)
\end{equation}

Now we only need an expression for $\Delta \vec{r}\hh$. We can find one in terms of $\Delta \vec{r}\hhco$, the change in the center of mass separation of \ce{H2} and \ce{CO}.
\begin{align}
\Delta \vec{r}\hhco &= {\left(\vec{r}\hh - \vec{r}\co \right)}_{f} - {\left(\vec{r}\hh - \vec{r}\co \right)}_{i} \\
&=\Delta \vec{r}\hh - \Delta \vec{r}\co \\
&=\Delta \vec{r}\hh \left(1 + \frac{m\hh}{m\co} \right)
\end{align}
where we have used \refeq{rcoASrhh} to eliminate $\Delta \vec{r}\co$. Continuing with $2T$ we insert this expression into \refeq{T2-1}, which yields:
\begin{align}
2T &= {\left( \frac{\Delta \vec{r}\hhco}{t}\right)}^2 \frac{m\hh}{ \left(1 + \frac{m\hh}{m\co} \right)} \\
&= {\left( \frac{\Delta \vec{r}\hhco}{t}\right)}^2 \mu\hhco
\end{align}
where $\mu\hhco$ is the reduced mass for \ce{H2} and \ce{CO}, defined in the usual way. Now our expression for the total length is:
\begin{align}
\ell_{Total} &= \int^{t}_{0} d\tau {\left( \frac{\norm{\Delta \vec{r}\hhco}}{t}\right)} \sqrt{ \mu\hhco} \\
&= \norm{\Delta \vec{r}\hhco} \sqrt{ \mu\hhco}
\end{align}

Using the appropriate masses from table \ref{tab:masses} gives us $\mu\hhco = 3427.5\;m_e$. We can estimate $\norm{\Delta \vec{r}\hhco}$ to be of order 10 $a_0$ because the MD trajectories were terminated when the centers were separated by 12 $a_0$. This gives us an estimate for the kinematic length of: $\ell_{Total} \approx 585\;a_0 \sqrt{m_{e}}$. This is at the very bottom of the distribution of total lengths for direct paths. This makes sense because we ignored many of the motions involved; modulo our guess for $\norm{\Delta \vec{r}\hhco}$, the estimate \emph{should} be a lower bound!

\paragraph{Vibrational Kinematic Length:} A similar analysis can be preformed for vibrational degrees of freedom. Using the expression for $T_{Vib}$ from \refeq{tvib} we have:
\begin{equation}
2T = \mu \dot{\vec{r}}^{\;2}
\end{equation}
Again using the assumption of constant velocity yields:
\begin{equation}
\dot{\vec{r}} = \frac{\Delta \vec{r}}{t}
\end{equation}
The expression is readily integrated to give:
\begin{equation}
\ell_{Vib} = \norm{\Delta \vec{r}\,} \sqrt{\mu}
\end{equation}

In the specific case of \ce{H2} vibration, we have $\mu = 918.56\;m_e$. We posit $\norm{\Delta \vec{r}\,}$ is of order 1 $a_0$ and therefore estimate the vibrational length as $\ell_{Vib} \approx 30\; a_0 \sqrt{m_{e}}$. This estimate is also fairly close to the lower bound of the distribution, which tails off around 40 $ a_0 \sqrt{m_{e}}$. Combining this result with the one for total length allows us to estimate the vibrational fraction to be: $\ell_{Vib}/\ell_{Total} \approx 30/585 = 0.05$, which is also within the tail of the distribution.

\bibliographystyle{JAmChemSoc}
\bibliography{../bibs/general,../bibs/roaming,../bibs/geodesics,../bibs/semiclassical,../bibs/self}

\end{document}
